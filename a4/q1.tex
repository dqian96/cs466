\documentclass{article}

\usepackage[utf8]{inputenc}
\usepackage{amsmath}
\usepackage{graphicx}
\usepackage{enumitem}
\usepackage[parfill]{parskip}
\graphicspath{ {.} }
\usepackage{listings}
\usepackage{xcolor}
\lstset { %
    language=C++,
    backgroundcolor=\color{black!5}, % set backgroundcolor
    basicstyle=\footnotesize,% basic font setting
}

\newcommand\floor[1]{\lfloor#1\rfloor}
\newcommand\ceil[1]{\lceil#1\rceil}

\newtheorem{theorem}{Theorem}

\usepackage{algorithm}
\usepackage[noend]{algpseudocode}

\title{CS 466: Algorithm Design and Analysis - Assignment 4}
\author{Qian, Yan Liang (David)}
\date{Date of submission: October 28, 2018}

\begin{document}
\newpage

\section{Question 1}


\textbf{Intuition}

Suppose we have a problem $X \in RP \land co$-$RPP$, then we know that there is a RPP algorithm $A$ and a co-RPP algorithm $A'$
for $X$. We will show that there is an algorithm $B$ that uses $A$ and $A'$ to solve $X$ in expected polynomial time,
with 0 probability of error, thus showing that $RPP \land co$-$RPP \subseteq ZPP$.

Suppose $x$ is an instance of $X$. Then, there are two cases. Either $x$ is a YES instance, or a NO instance.

\begin{itemize}
    \item $x$ is YES: By definition $A(x) = YES$ with probability $\geq \dfrac{1}{2}$, and NO otherwise. Note that $YES$
        responses by $RPP$ algorithms are always correct. Thus, we can keep on running $A(x)$ until $A(x) = YES$, in
        which case we have the correct answer and we terminate the algorithm.
    \item $x$ is NO: By definition $A'(x) = NO$ with probability $\geq \dfrac{1}{2}$, and YES otherwise. Note that $NO$
        responses by $co$-$RPP$ algorithms are always correct. Thus, we can keep on running $A'(x)$ until $A'(x) = NO$, in
        which case we have the correct answer and we terminate the algorithm.
\end{itemize}

In the absence of either algorithm, we will not be able to be fully confident in our answer in one of the cases. For
example, if given only $A$, then in the case where $x$ is NO, $A(x) = NO$ always. But we cannot be sure if $x$ is indeed
NO as if $x$ were a YES instance, then $NO$ could be indefinitely outputted anyway.

\textbf{Algorithm}

Run $A$ and $A'$ one after the other. Repeat running the algorithms until $A(x) = YES$ or $A'(x) = NO$, in which case we
return $YES$ and $NO$ respectively.

\begin{algorithm}
    \caption{ZPP Algorithm}\label{euclid}
    \begin{algorithmic}[1]
        \Procedure{B}{x, A, A'}
        \While {true}
            \If{$A(x) = YES$}
                \State \Return $YES$
            \EndIf
            \If{$A'(x) = NO$}
                \State \Return $NO$
            \EndIf
    \end{algorithmic}
\end{algorithm}

\textbf{Analysis}

It's easy to see that this algorithm is correct. Since we will only ever terminate if $A(x) = YES$ or $A'(x) = NO$, we
will always return the correct answer since both $A$ and $A'$ cannot be wrong on $YES$ and $NO$ outputs,
respectively. The algorithm will produce the correct output on $YES$ and $NO$ problem instances, or run indefinitely.
However, we will show that the expected number of runs in both $YES$ and $NO$ instances is a constant.

Let's calculate the expected number of runs of either algorithm in each case. Let's look at the YES case first. Since we
run $A$ and $A'$ repeatedly one after the other, $A(x) = YES$ will be either on the first, third, fifth, etc. run(s).
The result is very similar in the NO case.  The only difference is that $A'(x) = NO$ could report NO either on the
second, fourth, or sixth run(s).  The summation for the expected number of runs is as follows (note that the probability
of $A(x) = YES$ or $A'(x) = NO$ for $YES$ and $NO$ instances respectively is at least $\dfrac{1}{2}$).


\begin{align*}
    E(\# runs \: x \: is \: YES) &\leq \sum_{i=0}^{\infty} \dfrac{2i + 1}{2^{i+1}} \\
                      &\leq \sum_{i=0}^{\infty} \dfrac{i}{2^{i}} + \dfrac{1}{2} \sum_{i=0}^{\infty} \dfrac{1}{2^{i}} \\
                      &\leq 2 + \dfrac{1}{2} \\
                      &\leq \dfrac{3}{2} \\
    E(\# runs \: x \: is \: NO)  &\leq \sum_{i=0}^{\infty} \dfrac{2i}{2^{i}} \\
                      &\leq 2 \sum_{i=0}^{\infty} \dfrac{i}{2^{i}} \\
                      &\leq 2 \times 2 \\
                      &\leq 4
\end{align*}

* Note that $\sum_{i=0}^{\infty} \dfrac{1}{2^i} = 1$ from the result in class. To determine $S = \sum_{i=0}^{\infty}
\dfrac{i}{2^i}$, let's first consider $\dfrac{S}{2} = S - \dfrac{S}{2} = \dfrac{1}{2} + \dfrac{1}{4} + \dfrac{1}{8} +
\ldots = \sum_{i=0}^{\infty} \dfrac{1}{2^i} = 1$. Thus $S = 2$.

Since both $A$ and $A'$ run in poly-time by definition, we know that the expected number of runs in either case is $\leq
4$. Thus, the expected run time of $B$ is $4 * O(poly)$, which is still poly-time.

Now, we have an algorithm $B$ that is always correct and runs in expected poly-time, so $B \subseteq ZPP$. Thus, $RPP
\land co$-$RPP \in ZPP$.

\end{document}
