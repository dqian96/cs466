\documentclass{article}

\usepackage[utf8]{inputenc}
\usepackage{amsmath}
\usepackage{graphicx}
\usepackage{enumitem}
\usepackage[parfill]{parskip}
\graphicspath{ {.} }
\usepackage{listings}
\usepackage{hyperref}
\usepackage{xcolor}
\lstset { %
    language=C++,
    backgroundcolor=\color{black!5}, % set backgroundcolor
    basicstyle=\footnotesize,% basic font setting
}

\newcommand\floor[1]{\lfloor#1\rfloor}
\newcommand\ceil[1]{\lceil#1\rceil}

\newtheorem{theorem}{Theorem}

\usepackage{algorithm}
\usepackage[noend]{algpseudocode}

\title{CS 466: Algorithm Design and Analysis - Assigment 4}
\author{Qian, Yan Liang (David)}
\date{Date of submission: October 28, 2018}

\begin{document}
\newpage

\section{Question 2}

a)

Prove $S_1 = S_2 \Rightarrow P(S_1) \equiv P(S_2)$.

By definition, the polynomials formed from $S_1 = \{e_1, e_2, \ldots, e_n \}$ and $S_2 = \{f_1, f_2, \ldots, f_n\}$ are
$P(S_1) = (x - e_1)(x - _2)\ldots(x - e_n)$ and $P(S_2) = (x - f_1)(x - f_2)\ldots(x - f_n)$. Now, since multiplication
is commutative, we can re-arrange factors of both $P(S_1)$ and $P(S_2)$ by increasing order of $e_i$ and $f_i$,
respectively. Now, $P(S_1) = (x - e_{s1})(x - e_{s2})\ldots(x - e_{sn})$ and $P(S_2) = (x - f_{s1})(x - f_{s2})\ldots(x -
f_{sk})$.  Now, since $S_1$ and $S_2$ contain the same numbers with the same multiplicities, then the elements in sorted
order must match index for index. Thus, we have $e_{si} = f_{si}$ for all $i$. Thus, the polynomials match exactly.

Prove $P(S_1) \equiv P(S_2) \Rightarrow S_1 = S_2$.

If two polynomials in $x$ are the same, then they must have the same coefficients when written in the form $a_n x^n +
\ldots + a_1 x^1 + a_0$ by definition. Now, if we have a polynomial with complex coefficients (all coefficients being
real in this case), then the polynomial is uniquely factored as a product of linear terms, each of which corresponds to
a specific root of the polynomial. This theorem is proved
\href{http://mathonline.wikidot.com/the-factorization-of-polynomials-with-complex-coefficients}{here}. Thus, since there
is only one unique factoring to linear terms, both $P(S_1)$ and $P(S_2)$ can only be factored into linear terms in the
same way (since they are the same polynomial). Thus, we observe that the linear factors are the same, and so must be the
roots and their multiplicities such that $e_i = f_i$ for all $i \in [1, \ldots, n]$ (can be re-ordered as such). Thus,
the multisets $S_1$ and $S_2$, which are created from $e_i$ and $f_i$ values in the linear factor forms of $P(S_1)$ and
$P(S_2)$, respectively, must be the same. Thus, $S_1 = S_2$.

Proof for theorem used: http://mathonline.wikidot.com/the-factorization-of-polynomials-with-complex-coefficients


b)

\textbf{Intuition}

From the lemma above, to determine if $S_1 = S_2$, it is equivalent to verify that $P(S_1) = P(S_2)$. To test if two
polynomials are identical, we can make use of Sharwtz-Zippel. Simply construct $Q = P(S_1) - P(S_2)$. Then, we find that
$Q(x) \neq 0 \Rightarrow P(S_1) \neq P(S_2)$ as the two different polynomials would report different values for the same
$x$. Furthermore, $Q(x) = 0 \Rightarrow P(S_1) = P(S_2) \lor x \in roots(Q)$. In case 1, we can be sure that $S_1 \neq
S_2$ and in case 2, there is ambiguity in the equality of the two sets.

\textbf{Algorithm}

Construct $P(S_1)$ and $P(S_2)$ for the two sets. Construct $Q = P(S_1) - P(S_2)$. We can represent these polynomials
using arrays containing $e_i$ for each factor. Create a set $S$ of size $2n$ from randomly drawn reals. Thus, the
probability of randomly drawing a root from $S$ is at most $\dfrac{1}{2}$ (by Sharwtz-Zippel). Draw $x$ from $S$. Check
if $Q(x) = 0$ or $Q(x) \neq 0$ and return $YES$ (possible error) and $NO$, respectively.

\begin{algorithm}
    \caption{Multiset Equality}\label{euclid}
    \begin{algorithmic}[1]
        \Procedure{MultisetEquality}{$S_1$, $S_2$}
        \If{$|S_1| \neq |S_2|$} \Return $NO$ \EndIf
        \State $P(S_1)$ \gets $[e for e \in S_1]$  \Comment rep poly as list of factors to be multiplied
        \State $P(S_2)$ \gets $[e for e \in S_2]$
        \State $d \gets |S_1|$
        \State $S \gets \Call{RandSet}{2d}$
        \State $x \gets S[\Call{rand}{1, 2d}]$     \Comment draw value from $S$ in constant time (model)
        \State $Q(x) \gets \Call{CalculatePoly}{P(S_1), P(S_2), x}$
        \If{$Q(x) \neq 0$}                         \Comment polynomials not equal
            \State \Return $NO$
        \EndIf
        \State \Return $YES$                        \Comment really a ``maybe''
    \end{algorithmic}

    \begin{algorithmic}[1]
        \Procedure{RandSet}{$n$}
        \State $A \gets [n]$
        \For{$i \in [1 \ldots n]$}
            \State $A[i] \gets \Call{rand}{-n, n}$
        \EndFor
        \State \Return $A$
    \end{algorithmic}

    \begin{algorithmic}[1]
        \Procedure{CalculatePoly}{$P_1, P_2, x$}
        \State $value_1 \gets 1$
        \State $value_2 \gets 1$
        \For{$i \in [1 \ldots |P_1|]$}
            \State $value_1 \gets value_1 \times (x - P_1[i])$
            \State $value_2 \gets value_2 \times (x - P_2[i])$
        \EndFor
        \State \Return $value_1 - value_2$
    \end{algorithmic}
\end{algorithm}


\textbf{Analysis}

First off, let's look at the runtime of the algorithm. Generating the arrays that store the two polynomials $P$ is
linear with respect to the number of elements $n$. Creating a random set of size $2d = 2n$ takes $O(2n)$ time since we
can draw random numbers within a range in constant time by the definition of our model of randomness. Calculating $Q$
simply takes $O(n)$ time as well as we iterate over the two lists of size $n$ at the same time, and all arithmetic is
constant time. Thus, the runtime of the algorithm is $O(n)$.

As a concern of practicality, we could do our computation with $mod$ given some small prime to prevent overflow, but
this could increase our probability of error. This would no doubt be a more practical algorithm, but as mentioned on
Piazza, we do not have to worry about this practicality concern in this assignment and just do computation over reals.

Now, we know that $Q$ is a single variable polynomial of total degree $n$. Now, in the case when $Q$ is identitically
equal to 0, then $Q(x) = 0$ for any $x$. Reporting $YES$ in this case would be correct as the two sets are equal.

Let's look at the case where $Q$ is not identitically 0. Following Sharwtz-Rappel, if we randomly draw $x$ from $|S|$,
then $Pr\{Q(x) = 0\} \leq \dfrac{n}{2n} = \dfrac{1}{2}$. With probability $\leq \dfrac{1}{2}$, our algorithm would
report $YES$ despite the fact that two multisets are not equal ($Q$ is not identitically 0). The error in this case is
at most $\dfrac{1}{2}$.

Thus, we have the following:
\begin{itemize}
    \item NO ($S_1 \neq S_2$): $Pr\{MultisetEquality(S_1, S_2) = NO\} \geq \dfrac{1}{2}$
    \item YES ($S_1 = S_2$): $Pr\{MultisetEquality(S_1, S_2) = NO\} = 0$
\end{itemize}

Thus, we have an algorithm that has one-sided error. Our algorithm will always report $NO$ correctly, and report $YES$
with an error of at most $\dfrac{1}{2}$.


\end{document}
