\documentclass{article}

\usepackage[utf8]{inputenc}
\usepackage{amsmath}
\usepackage{graphicx}
\usepackage{enumitem}
\usepackage[parfill]{parskip}
\graphicspath{ {.} }
\usepackage{listings}
\usepackage{xcolor}
\lstset { %
    language=Python,
    backgroundcolor=\color{black!5}, % set backgroundcolor
    basicstyle=\footnotesize,% basic font setting
}

\newcommand\floor[1]{\lfloor#1\rfloor}
\newcommand\ceil[1]{\lceil#1\rceil}

\newtheorem{theorem}{Theorem}

\usepackage{algorithm}
\usepackage[noend]{algpseudocode}

\title{CS 466: Algorithm Design and Analysis - Assignment 6}
\author{Qian, Yan Liang (David)}
\date{Date of submission: November 13, 2018}

\begin{document}
\newpage

\section{Question 2}

a)

Let's first prove that $b_j + b_{j+1} > 1$. Since the algorithm is online, suppose that we are at the point where bin
$j$ is still open and we have not created bin $j + 1$ yet. Suppose that an item $s_k$ comes such that it cannot fit in bin
$j$. This means that bin $j$ should be closed with a total amount of $b_j$ (total sum of items in bin $j$). Now, the
only way such that $s_k$ could not fit into bin $j$ is if $s_k > 1 - b_j$ (more than the remaining space is necessary).
We must put $s_k$ into bin $j + 1$ instead. Now, we have $b_{j + 1} \geq s_k > 1 - b_j$. Adding $b_j$ to this
inequality, we have $b_j + b_{j + 1} > 1 - b_j + b_j > 1$. Thus, our initial claim is proven.

Now, let's suppose we have $m$ bins after the last item is inserted. Let's find a strict lower bound for $\sum b_i$.
Break it into two cases where $m$ is odd or $m$ is even.  Suppose $m$ is even. Observe that $b_i + b_{i + 1} > 1$ for
all $i \in \{1, 3, 5, m-1\}$ by the above proof. By grouping subsequent bins into $\dfrac{m}{2}$ pairs, where the sum of
each pair is more than 1, we establish that $b_1 + b_2 + \ldots + b_{m-1} + b_{m} > 1 + 1 + \ldots + 1 > \dfrac{m}{2} >
\dfrac{m-1}{2}$. Suppose that $m$ is odd instead. If this were the case, then we can once again establish that $b_1 +
\ldots + b_m > \sum_{i \in \{1, 3, m-2\}} b_i + b_{i + 1} > \dfrac{m-1}{2}$ by ignoring the value of last bin $b_{m}$
and grouping the even number of $m-1$ bins into subsequent pairs, which yields $\dfrac{m-1}{2}$ pairs, whose sums are
more than 1 (each pair). These two cases tell us that $b_1 + b_2 + \ldots + b_m > \dfrac{m-1}{2}$ no matter the $m$.

Observe that $\sum b_j \leq OPT$ as the sum of all items/bin values represents an ``ideal'' packing scenario in which
there is no fragmentation, so the number of bins required in this best, idealized case is at least $\lceil \sum b_j
\rceil$. Thus, $OPT \geq \lceil \sum b_j \rceil$. In all other cases where there is fragmentation, the optimal solution
must still be greater than or equal to $\lceil \sum b_j \rceil$ as this is the theoretical minimum for storing the total
value of items. Thus $OPT \geq \lceil \sum b_j \rceil \geq \sum b_j$. (This result was also stated in class.) Since
$\dfrac{m-1}{2} < \sum b_j \leq OPT$, we have $\dfrac{m-1}{2} < OPT$, and so $m < 2OPT + 1$. Thus, the algorithm uses at
most $2OPT + 1$ bins.

\end{document}
