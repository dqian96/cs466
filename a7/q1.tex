\documentclass{article}

\usepackage[utf8]{inputenc}
\usepackage{amsmath}
\usepackage{graphicx}
\usepackage{enumitem}
\usepackage[parfill]{parskip}
\graphicspath{ {.} }
\usepackage{listings}
\usepackage{xcolor}
\lstset { %
    language=C++,
    backgroundcolor=\color{black!5}, % set backgroundcolor
    basicstyle=\footnotesize,% basic font setting
}

\newcommand\floor[1]{\lfloor#1\rfloor}
\newcommand\ceil[1]{\lceil#1\rceil}

\newtheorem{theorem}{Theorem}

\usepackage{algorithm}
\usepackage[noend]{algpseudocode}

\title{CS 466: Algorithm Design and Analysis - Assignment 6}
\author{Qian, Yan Liang (David)}
\date{Date of submission: November 13, 2018}

\begin{document}
\newpage

\section{Question 1}

a)

Yes. A poly-time constant factor approximation for the Independent Set (IS) problem does give a poly-time constant
factor approximation for the Clique (CQ) problem.

First we assert that if $U$ is a maximal independent set in $G'$ (complement of the graph), then $U$ is also the maximal
clique in $G$. Hence $OPT_{IS}(G') = OPT_{CQ}(G)$. This can be seen by the question statement: ``$U$ is an independent
set in $G$ $\iff$ $U$ is a clique in $G'$''. This equivalence indicates the largest IS in $G'$ is the largest CQ in $G$
as there is a one-to-one mapping (largest clique corresponds to an IS and vice versa).

Suppose we had a poly-time approximation algorithm for IS. Now, let's see if we can use the approximation algorithm for
IS to find a constant factor approximation algorithm for CQ given some graph $G$. Simply run the IS approximation
algorithm on $G'$ (poly-time operation) to get an IS $U$ for $G'$ with the bound $|U| \geq (1 -
\epsilon)|OPT_{IS}(G')|$. Since $U$ would then be completely connected in $G$, it is also an approximation for the CQ
problem. By the equality of the optimal solutions in the above paragraph, we would have this bound $|U| \geq (1 -
\epsilon)|OPT_{CQ}(G)|$. Thus, we get a poly-time constant factor approximation for the clique problem given the IS
approximation algorithm.


b)

No. A poly-time constant factor approximation for the Independent Set (IS) problem does not give a poly-time constant
factor approximation for the Vertex Cover (VC) problem.

First, we assert that if $U$ is a maximal independent set then $V - U$ is a minimal vertex cover. Hence $|V| =
|OPT_{IS}| + |OPT_{VC}|$. This can be seen by the question statement: ``$U \subseteq V$ is an IS $\iff$ $V-U$ is a VC''.
If $U'$ is the largest IS with a corresponding VC $V - U'$, then because there are no IS's bigger than $U'$, any smaller
$V - U$ is not possible since $U'$ is maximal. Thus, $V - U'$ is a minimal cover.

Suppose we had a poly-time approximation algorithm for IS. Then, we get the bound $|U'| \geq (1 - \epsilon)|OPT_{IS}|$.
Observe the following:

\begin{align*}
    |V| &= |V| \\
    |V| - |U'| &\leq |V| - (1 - \epsilon)|OPT_{IS}| \\
    |V| - |U'| &\leq |OPT_{IS}| + |OPT_{VC}| - (1 - \epsilon)|OPT_{IS}| \\
    |V| - |U'| &\leq |OPT_{IS}| + |OPT_{VC}| - |OPT_{IS}| + \epsilon |OPT_{IS}| \\
    |V| - |U'| &\leq |OPT_{VC}| + \epsilon |OPT_{IS}| \\
    |V| - |U'| &\leq \begin{cases}
        (1 + \epsilon)|OPT_{VC}| & \text{if} \: |OPT_{VC}| \geq |OPT_{IS}| \\
        |OPT_{VC}| + \epsilon |OPT_{IS}| & \text{else}
    \end{cases}
\end{align*}

From the above, it is clear that the approximation depends on the relative sizes of $|OPT_{VC}|$ and $|OPT_{IS}|$. Thus,
for some $\epsilon$, you would get a constant approx factor if the minimal VC is bigger than or equal to the maximal IS.
However, if the maximal IS is bigger than the minimal VC, the approximation factor/bound can be violated as $|OPT_{VC}|
+ \epsilon |OPT_{IS}| > (1 + \epsilon)|OPT_{VC}|$ (this is a reachable, tight bound for this case).  Thus, a constant
factor approximation for VC is not possible.

\end{document}
