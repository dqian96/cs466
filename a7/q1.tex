\documentclass{article}

\usepackage[utf8]{inputenc}
\usepackage{amsmath}
\usepackage{graphicx}
\usepackage{enumitem}
\usepackage[parfill]{parskip}
\graphicspath{ {.} }
\usepackage{listings}
\usepackage{xcolor}
\lstset { %
    language=C++,
    backgroundcolor=\color{black!5}, % set backgroundcolor
    basicstyle=\footnotesize,% basic font setting
}

\newcommand\floor[1]{\lfloor#1\rfloor}
\newcommand\ceil[1]{\lceil#1\rceil}

\newtheorem{theorem}{Theorem}

\usepackage{algorithm}
\usepackage[noend]{algpseudocode}

\title{CS 466: Algorithm Design and Analysis - Assignment 6}
\author{Qian, Yan Liang (David)}
\date{Date of submission: November 13, 2018}

\begin{document}
\newpage

\section{Question 1}

a)

Prove that $G$ has a clique of size $t$ $\iff$ $G^k$ has a clique of size $t^k$.

($\Rightarrow$) Let $C$ be a clique of size $t$ in $G$. Denote the vertices in the clique as $c_i \in C$, for all
integer $i \in [1, t]$. Let $C'$ denote the subset of vertices in $G^k$ where for each vertex $(v_1, \ldots, v_k) \in
C', v_i \in C$.  Now, for any two vertices $(u_1, \ldots, u_k)$ and $(v_1, \ldots, v_k)$ in $C'$, we see that for all
$i$, either $u_i = v_i$ or there must be an edge between $u_i$ and $v_i$ in $G$ as both $u_i, v_i \in C$ by construction
of $C'$. Thus, all vertices in $C'$ are completely connected to each other in $G^k$, meaning that $C'$ is a clique.  To
calculate the size of $C'$, observe that $C'$ enumerates all $k$-tuples where each entry $v_i$ is from $C$.  Since there
are $|C| = t$ possible values for any entry, there are $t^k$ possible combinations for tuples of size $k$. Thus, we have
$|C'| = t^k$. Thus, $C'$ is a clique of size $t^k$ that exists in $G^k$.

($\Leftarrow$) Following from the same conventions as above, $C'$ is a clique of size $t^k$ in $G^k$. This must mean
that there are $t^k$ distinct vertices in $C'$, which must mean that there are $t^k$ distinct combinations of $(v_1,
\ldots, v_k) \in C'$. To get $t^k$ distinct combinations, you need at least $t$ distinct elements for at least one index
of the $k$-tuples in $C'$ (i.e. suppose you have $t-1$ distinct elements, then the most number of unique $k$-tuples you can
make is $(t-1)^k$).  Since all vertices are completely connected in $C'$, we know that for the index that has at least
$t$ distinct possible values/vertices, there must be an edge in $G$ between each pair of vertices/values for that index
- otherwise there wouldn't be an edge between the $k$-tuple vertices in $C'$. Thus, there are at least $t$ vertices that
are completely connected to each other in $G$, meaning there is a clique of size $t$ in $G$.


b)

\textbf{Intuition}

Suppose the maximum clique in $G$ has size $t$. ``Raising'' $G$ to some $k$ will map this clique to a larger clique of
size $t^k$. A constant factor approximation algorithm will be able to find a clique that is a subset of these vertices,
say of size $\dfrac{t^k}{2}$. We know that this clique has $\dfrac{t^k}{2}$ unique combinations of $k$-tuples.
Correspondingly, for some index of the clique of $k$-tuples, there must be at least $\sqrt[k]{\dfrac{t^k}{2}}$ distinct
values for $\dfrac{t^k}{2}$ combinations to be possible.  Since the clique is completely connected, we know that those
$\sqrt[k]{\dfrac{t^k}{2}}$ vertices are completely connected in $G$ and must form a clique. Note that the clique
found in $G$ approximates to $\dfrac{t}{\sqrt[k]{2}}$ - this means as $k \rightarrow \infty$, we have
$\dfrac{t}{\sqrt[k]{2}} \rightarrow t$.

\textbf{Proof}

Given $G$, let's find the maximum cover in $G^k$ for some $k$. Let $OPT$ be the maximum cover in $G$ and $OPT^k$ be the
maximum cover in $G^k$.We will determine this $k$ later (honestly, what's with determining everything retrospectively).
Run $A$ on $G^k$ to find a cover $C^k$. We know that $|C^k| \geq \dfrac{1}{2}|OPT^k|$ because the approximation factor of
$A$ is $\dfrac{1}{2}$. Let $m \geq \dfrac{1}{2}$ be the actual approximation achieved - thus, $|C^k| = m |OPT^k|$. Now,
going through the $k$-tuples $\in C^k$, we have at least $|C^k| = t^k$ possible combinations for some $t$. By the
previous part, we know that there is a clique in $G$ of size $t$, which is an approximation of the maximum clique. Let
$t = (1 - \epsilon)|OPT|$. Thus, we have $((1 - \epsilon)|OPT|)^k = |C^k| = m |OPT^k|$.

Note: In practice, we know that at least one index for the $k$-tuples in $C^k$ have $\lceil t \rceil$ distinct values
(otherwise, $t^k$ combinations are not possible for $C^k$).

Observe that $|OPT|^k = |OPT^k|$ as the largest clique in $G$ maps to a clique in $G^k$ and vice
versa by the above question. Now, let's re-arrange:

\begin{align*}
    ((1 - \epsilon)|OPT|)^k &= m|OPT^k| \\
    (1 - \epsilon)^k |OPT|^k &= m|OPT^k| \\
    (1 - \epsilon)^k  &= m \\
    k \log(1 - \epsilon)  &= \log m \\
    k &=\dfrac{\log m}{\log(1 - \epsilon)} \\
    k &= \dfrac{-1}{\log(1 - \epsilon)} && \text{worst case $m = \dfrac{1}{2}$; greater $m$ requires smaller $k$}
\end{align*}

Thus, for some given $\epsilon$ such that we get an approximation clique of size $(1 - \epsilon)|OPT|$ in $G$, we need
to at least $k = \dfrac{-1}{\log(1- \epsilon)}$.


\textbf{Algorithm}

Based on the above proof, we first determine the necessary $k = \lceil \dfrac{-1}{\log(1 - \epsilon)} \rceil$ based on
the given $\epsilon$. Then, we create $G^k$. We run $A$ on $G^k$ and then for the clique $C^k$ returned, we look through
every index to find the largest list of values/vertices possible for that index. The maximal list will be $C$ such that
$|C| \geq (1 - \epsilon)|OPT|$.

\begin{algorithm}
    \caption{PTAS for Clique}\label{euclid}
    \begin{algorithmic}[1]
        \Procedure{PTASClique}{$G = (V, E)$, $\epsilon$}
        \State $k \gets \lceil \dfrac{-1}{\log (1 - \epsilon)} \rceil$
        \State create graph $G^k$ brute-force
        \State $C^k \gets \Call{A}{G^k}$
        \State $C \gets \emptyset$
        \For{$i \in [1, \ldots,  |C^k|]$}
            \State $S \gets$ distinct vertices in $C^k$ for index $i$
            \If{$|S| \geq |C|$}
                \State $C \gets S$
            \EndIf
        \EndFor
        \State \Return $C$
    \end{algorithmic}
\end{algorithm}

\textbf{Analysis}

The bottle-neck of this algorithm is generating $G^k$ for $k = \lceil \dfrac{-1}{\log ( 1 - \epsilon )} \rceil$. It's
clear that there are $n^k$ vertices ($n$ choices for each of the $k$ indices in the tuple). There are at most
$O(n^{2k})$ edges. A brute force graph construction of $G^k$ in which you iterate over every vertex/$k$-tuple, for each
index and compare it with all the other vertices is bounded by $O(k n^{2k}) \in O(n^{3k})$. Running $A$ on this graph is
$O(p(n^{k}))$.  Iterating through the maximal clique in $G^k$ to find the maximal list of vertices for each index is no
more than $O(n^{3k})$ (high upper bound than necessary because it's irrelevant/not the bottleneck). Thus, we see that
the total runtime of the algorithm is $O(n^{\dfrac{-3}{\log(1 - \epsilon)}} + p(n^{\dfrac{-1}{\log (1- \epsilon)}}))$.


\end{document}
