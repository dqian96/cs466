\documentclass{article}

\usepackage[utf8]{inputenc}
\usepackage{amsmath}
\usepackage{graphicx}
\usepackage{enumitem}
\usepackage[parfill]{parskip}
\graphicspath{ {.} }
\usepackage{listings}
\usepackage{xcolor}
\lstset { %
    language=C++,
    backgroundcolor=\color{black!5}, % set backgroundcolor
    basicstyle=\footnotesize,% basic font setting
}

\newcommand\floor[1]{\lfloor#1\rfloor}
\newcommand\ceil[1]{\lceil#1\rceil}

\newtheorem{theorem}{Theorem}

\usepackage{algorithm}
\usepackage[noend]{algpseudocode}

\title{CS 466: Algorithm Design and Analysis - Assigment 1}
\author{Qian, Yan Liang (David)}
\date{Date of submission: September 20, 2018}

\begin{document}
\newpage

\section{Question 1}

[10 marks] Suppose you have a splay tree with keys $1 \ldots n$ and you access the keys sequentially,
i.e. you search for $1, 2, \ldots, n$, in that order, performing a splay on each one.

a) [2 marks] What is the structure of the final splay tree? Justify your answer.





\newline

\textbf{Answer:} We will define potential to be the number of elements in the array.
Thus, when the array has $k$ elements, $\Phi = k$.
This guarantees that $\Phi_i \geq 0$ for all $i$, confirming that it is a valid potential.
Furthermore, we see that this allows the initial potential to be $\Phi_0 = 0$ as initially, we start off with
an empty array.
\newline

b) Prove that the final potential is $\geq$ initial potential.
\newline

\textbf{Answer:} Initially, we start off with an empty array - there are 0 elements in the array. Thus, the initial
potential is 0.
The number of elements stored in the array can only ever be a non-negative number. Thus, the final array contains
\textit{at least} 0 elements, and so the final potential is at least 0, which means it's at least the initial potential.
\newline

c) Conclude that the amortized cost of each operation is $O(1)$.
\newline

\textbf{Answer:} In class, we were given the following theorem:

\begin{theorem}
    If final potential $\geq$ initial potential, then amortized cost $\leq$ max charge.
\end{theorem}

From part b), we know that final potential $\geq$ initial potential. Thus, we can apply this theorem to the two
operations.

For the add operation, we are given the constant $charge = 2$. Thus, by the theorem, we know that amortized cost $\leq
2$. Since 2 is a constant, we know that amortized cost $\in O(1)$.

For the empty operation, we are given the constant $charge = 1$. Thus, by the theorem, we know that amortized cost $\leq
1$. Since 1 is a constant, we know that amortized cost $\in O(1)$.
\newline


\end{document}

