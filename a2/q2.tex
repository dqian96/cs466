\documentclass{article}

\usepackage[utf8]{inputenc}
\usepackage{amsmath}
\usepackage{graphicx}
\usepackage{enumitem}
\usepackage[parfill]{parskip}
\graphicspath{ {.} }
\usepackage{listings}
\usepackage{xcolor}
\lstset { %
    language=C++,
    backgroundcolor=\color{black!5}, % set backgroundcolor
    basicstyle=\footnotesize,% basic font setting
}

\newcommand\floor[1]{\lfloor#1\rfloor}
\newcommand\ceil[1]{\lceil#1\rceil}

\newtheorem{theorem}{Theorem}

\usepackage{algorithm}
\usepackage[noend]{algpseudocode}

\title{CS 466: Algorithm Design and Analysis - Assigment 1}
\author{Qian, Yan Liang (David)}
\date{Date of submission: September 20, 2018}

\begin{document}
\newpage

\section{Question 2}


[3 marks] Show that this is impossible using the fact that sorting a list of n elements requires $\Omega (n \log n)$
time.

\textbf{Answer:}

Suppose we are given an array of $n$ numbers $A$. Let's now insert these numbers into a binomial heap. Given that the
amortized cost for insert is $O(1)$, then the cost of inserting these $n$ numbers should be $O(n)$. Now, let's
call delete-min to delete $n$ numbers from the heap and re-insert them into the array in the order that we pop them.
Since the amortized cost is constant for delete-min, deleting $n$ elements will cost $O(n)$ total. Since we pop the
numbers from the least to greatest, our array is now in sorted order. We now sorted the array in $O(n) + O(n) \in O(n)$
time, which is not possible given that the lower bound for comparision based sorting is $\Omega (n \log n)$. Thus, these
operations cannot be both amortized constant time.


ii) [4 marks] Explain precisely what is wrong with M.A. Zing’s argument.

\textbf{Answer:}

The problem with M.A Zing's argument is that his calculation of the amortized merge operation is based on merging
binomial heaps of the same size. Every insert operation operation increases the size of the binomial heap by 1, and the
merge is performed with 1 element instead. This increases the entropy in the data structure.

For delete min, we additionally have to find the minimum element, which is itself $\log n$, so it requires more than
just the cost of the merge.


\end{document}
