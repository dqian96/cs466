\documentclass{article}

\usepackage[utf8]{inputenc}
\usepackage{amsmath}
\usepackage{graphicx}
\usepackage{enumitem}
\usepackage[parfill]{parskip}
\graphicspath{ {.} }
\usepackage{listings}
\usepackage{xcolor}
\lstset { %
    language=C++,
    backgroundcolor=\color{black!5}, % set backgroundcolor
    basicstyle=\footnotesize,% basic font setting
}

\newcommand\floor[1]{\lfloor#1\rfloor}
\newcommand\ceil[1]{\lceil#1\rceil}

\newtheorem{theorem}{Theorem}

\usepackage{algorithm}
\usepackage[noend]{algpseudocode}

\title{CS 466: Algorithm Design and Analysis - Assigment 3}
\author{Qian, Yan Liang (David)}
\date{Date of submission: October 15, 2018}

\begin{document}
\newpage

\section{Question 1}

\textit{Note on notation: we will represent the set with name $s$ as $S$.}

\textbf{Intuition}

Let's model this problem with the help of disjoint sets. Given the array $A[1, \ldots n]$, we will
partition indices $1, \ldot n$ into disjoint sets by their solution to operation C. That is, $\forall i$ such that
$OperationC(i) = k \iff i \in K$. That is, the indices with the same solution for operation C are in the same set $K$ with "name" $k$. Note that all the
indices in the set have to be a consecutive sequence. If it's not a consecutive sequence, then there exists such index
$j$ in between the indices in $K$ such that $A[j] = 0$, and so $\exists h \in K,Operationc(h) \neq k$, violating our
definition. Initially, since $A$ contains only 0's, then we have $n$ disjoint sets, one for each index. On setting $A[i]
= 1$, we can see that $OperationC(i) = OperationC(i+1)$ - that is, since $A[i]$ is no longer a zero, the leftmost zero
for $i$ is now the same as the leftmost zero for $i + 1$. According to our definition above, $i$ and $i+1$ must now be
in the same set. We can accomplish this by an $Union$ operation between the sets $i$ and $i+1$ belong to. When we need
to recover which set $i$ is in, that is the solution to operation C, we can simply call $Find(i)$. Something interesting
to note is that $A[i]$ will become 1 only once, and so we only union $i$ and $i+1$ once. Given these requirements, let's
keep track of these disjoint sets in another array $U$ (in addition to $A$ containing the 0-1 data).
To implement $Union$ and $Find$ for $U$, we use the same algorithm taught in class using tree linking and path
compression. Explicitly:

\begin{enumerate}
    \item $OperationA(i)$: Call $Union(i, k)$, where $k$ is the set name of the set that $i + 1$ belongs to. This can be
        found in constant time with the help of supplementary array $Q$ described below. If $K$ is smaller than $I$,
        $Union$ will merge $K$ into $I$. However, $OperationC$ should return $k$, not $i$. Thus, we will change the name
        of the union'd set from $i$ to $k$ via a swap on the previous root nodes $i$ and $k$ after the union, thus
        making $k$ the root of the union'ed set. This is done in constant time.
    \item $OperationB(i)$: simply return $A[i]$
    \item $OperatonC(i)$: $Find(i)$.
\end{enumerate}

In addition to the union find data structure, we will also keep another array $Q$ that we will we use to store
additional information. The goal of this data structure is to determine the root/set of the leftmost element (smallest
index) in some set $I$ in constant time. Note that in this particular application of union find, all sets are within a
contiguous section of the array and all "unions" will only occur between the head/root (since we flip $A[i]$ from 0 to
1) of a subarray/set and the tail (leftmost child) of another subarray/set. Let the following rules define array $Q$ of
size $n$.

\begin{enumerate}
    \item Property: Let $i$ be the name of a set that still exists (has not been absorbed into a larger set), then the following
        is true: $\exists k \geq 0$, $(Q[i] = k  \iff Q[i-k] = -k) \land j \in [i-k, \ldot i], j \in I$. This also means
        that $OperationC(j) = i$ from our definition above. This means that everything in the contiguous subarray
        from $i-k$ to $i$ is in set $I$, thereby meaning the set $I$ has size $k + 1$ and that the leftmost right
        zero for all indices in this range is at index $i$.
    \item Initially $Q[i] = 0 \forall i \in [1, \ldot n]$.
    \item We need to maintain this property when we union disjoint sets. Note that a find will not impact this array at
        all since set membership does not change. We can only ever union sets $i$ and the set that $i+1$ belongs to
        (i.e. disjoint sets when we flip $A[i] = 0$ to 1). In other words, in $Q$, the two sets are represented by two
        contiguous subarrays "next to each other". With our property above, we know that when we want to flip $A[i] =
        1$, index $i+1$ belongs to set $K$ such that $k= i + 1 - A[i + 1]$. Now there are two cases when we call
        $Union(i, k)$. Note that both of these take $O(1)$ time as it is just array lookup and changing values.
        \begin{enumerate}
            \item If $|K| \geq |I|$, then we set $A[i - A[i]] -= A[k] + 1$ and we set $A[k] += 1 - A[i - A[i]]$. Note
                that this happens simultaneously (for example we store $A[k]$ in a temp variable so we increment using the old
                value than the new value).
            \item If $|K| < |I|$, we do the exact same as above except that after we do the union, we notice that since
                the larger of the sets was $I$, the elements in the contiguous subarray $[i - A[i] \ldots k]$ no longer
                belong in $K$, but in $I$ instead! This violates our property above! So, to account for this we change
                the name of set $I$ to $k$ instead. This can be easily done by keeping track of the root node of $K$
                before linking and swapping it with the root node of the union'd set (which contains $i$) after linking.
                Thus, the union'd set $K$ now has a root of $k$. This keeps our property.
        \end{enumerate}
\end{enumerate}


\textbf{Analysis}

\begin{enumerate}
    \item Operation A: Finding the set name of $i + 1$ is just constant time as it is array index lookup specified
        above. $Union$ given the roots of two sets is just $O(1)$ as it is just adjusting pointers. Don't forget we also
        update $Q$ on $Union$. However, this simply consists of array indexing for $i$ and $i + 1$, which is $O(1)$.
        This results in a total runtime of $O(1)$.
    \item Operation B: $O(1)$ as this is an array lookup.
    \item Operation C: This is just amortized $O(\alpha (m. n))$ from the results in class.
\end{enumerate}


\textbf{Algorithm}


\begin{algorithm}
    \caption{Operations}\label{euclid}
    \begin{algorithmic}[1]
        \Procedure{OperationA}{A, i, Q, U}
        \State $k \gets i + 1 - Q[i + 1]$  \Comment{root of the second set (i + 1)}
        \State $swap \gets False$
        \If{$A[k] < A[i]$}
            \State $swap \gets True$
        \EndIf
        \State $temp \gets A[i - A[i]]$
        \State $A[i - A[i]] \gets A[i - A[i]] - A[k] - 1$
        \State $A[k] \gets A[k] - A[temp] + 1$
        \State $\Call{U.Union}{i, k}$
        \If{$swap$}
            \State $\Call{ChangeName}{i, k}$ \Comment change name of set $I$ to $k$ by a swap between the previous roots
        \EndIf
        \EndProcedure
    \end{algorithmic}


    \begin{algorithmic}[1]
        \Procedure{OperationB}{A, i, Q, U}
        \State \Return $A[i]$
        \EndProcedure
    \end{algorithmic}

    \begin{algorithmic}[1]
        \Procedure{OperationC}{A, i, Q, U}
        \State \Return $ \Call{U.Find}{i}$
        \EndProcedure
    \end{algorithmic}
\end{algorithm}


\end{document}
