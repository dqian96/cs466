\documentclass{article}

\usepackage[utf8]{inputenc}
\usepackage{amsmath}
\usepackage{graphicx}
\usepackage{enumitem}
\usepackage[parfill]{parskip}
\graphicspath{ {.} }
\usepackage{listings}
\usepackage{xcolor}
\lstset { %
    language=C++,
    backgroundcolor=\color{black!5}, % set backgroundcolor
    basicstyle=\footnotesize,% basic font setting
}

\newcommand\floor[1]{\lfloor#1\rfloor}
\newcommand\ceil[1]{\lceil#1\rceil}

\newtheorem{theorem}{Theorem}

\usepackage{algorithm}
\usepackage[noend]{algpseudocode}

\title{CS 466: Algorithm Design and Analysis - Assignment 8}
\author{Qian, Yan Liang (David)}
\date{Date of submission: November 27, 2018}

\begin{document}
\newpage

\section{Question 1}

Prove that this stupid caching algorithm is $k$-competitive.

Let $\nu$ be a sequence of page requests. Like done so in class, we can divide the request sequence into $p$ phases
where a phase stops just before we see $k + 1$ distinct pages.

Let's analyze the cost of the stupid algorithm by using the concept of ``phases''. According to the stupid algorithm,
the cache will be completely purged if it is full. Thus, after each phase except the last, the cache becomes full (after
ingesting $k$ distinct pages), and will be purged with a cost of $k$ at the start of the next phase (where a new page is
requested). The cost for the last phase is 0 since no pages are evicted.

The optimal algorithm, on the other hand, will evict at least 1 page per phase, except for the last one where it might
do at least 0 work.

Thus, the ratio of the work done by the stupid algorithm $A$ for the first $p - 1$ phases compared to the optimal for
the whole sequence is $\dfrac{A(\nu_{p-1})}{OPT(\nu)} \leq k$. Adding the work $c$ done by $A$ for the last phase, we
get the bound $A(\nu_{p-1}) + c = A(\nu) \leq k OPT(\nu) + c$. Since $c = 0$ as no pages are evicted by the last phase by
the stupid algorithm, we get $A(\nu) \leq k OPT(\nu)$, proving that our stupid algorithm is $k$-competitive.

\end{document}
