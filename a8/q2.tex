\documentclass{article}

\usepackage[utf8]{inputenc}
\usepackage{amsmath}
\usepackage{graphicx}
\usepackage{enumitem}
\usepackage[parfill]{parskip}
\graphicspath{ {.} }
\usepackage{listings}
\usepackage{xcolor}
\lstset { %
    language=Python,
    backgroundcolor=\color{black!5}, % set backgroundcolor
    basicstyle=\footnotesize,% basic font setting
}


\newtheorem{theorem}{Theorem}

\usepackage{algorithm}
\usepackage[noend]{algpseudocode}

\usepackage{mathtools}
\DeclarePairedDelimiter\floor{\lfloor}{\rfloor}


\title{CS 466: Algorithm Design and Analysis - Assignment 6}
\author{Qian, Yan Liang (David)}
\date{Date of submission: November 13, 2018}

\begin{document}
\newpage

\section{Question 2}

a)

We will prove that this deterministic algorithm has competitive ratio $\dfrac{1}{\sqrt{B}}$.

Consider case 1, $M \geq T$.  In this case, the algorithm will always accept a bid $D$ such that $T = \sqrt{B} \leq D
\leq M$ as such a bid will always appear in the sequence and the algorithm will accept the first one it sees. The
optimal algorithm would pick $M$. Then, the ratio between this algorithm and the OPT is $\dfrac{ALG}{OPT} =
\dfrac{D}{M}$. In the worst case, we minimize this ratio by considering the largest $M$ and smallest $D$ to get a lower
bound of $\dfrac{\sqrt{B}}{B} = \dfrac{1}{\sqrt{B}}$. \\

Consider case 2, $M < T$. In this case, the algorithm will pick the last bid $D$ such that $1 \leq D \leq M < T = \sqrt{B}$.
The optimal algorithm would pick $M < \sqrt{B}$. Then, the ratio between this algorithm and OPT is $\dfrac{ALG}{OPT} =
\dfrac{D}{M}$. In the worst case, we minimize the ratio by considering the largest $M$ and the smallest $D$ to get a
lower bound of $\dfrac{1}{\sqrt{B} - 1} > \dfrac{1}{\sqrt{B}}$. Take $\dfrac{1}{\sqrt{B}}$ as the lower bound. (Note
that in the case where $B = 1$, then $\dfrac{1}{\sqrt{B}} = 1$ still satisfies as the lower bound). \\

Thus, in both possible cases, we see that the ratio between the algorithm's performance to that of the optimal algorithm
is $\dfrac{ALG}{OPT} \geq \dfrac{1}{\sqrt{B}}$. Thus, $ALG \geq \dfrac{1}{\sqrt{B}} OPT$ and the algorithm is
$\dfrac{1}{\sqrt{B}}$-competitive.



b)

We will prove that the expected competitive ratio is at least $\dfrac{1}{2 \log B}$. Consider $M$ such that it is $2^k
\leq M < 2^{k+1}$ for any $k \in [1 - 1, 2 - 1, 3 - 1, \ldots \floor{\log B + 1} - 1]$. Now, we can look at two cases
based on the randomly chosen threshold. For a given $k$, either $T = 2^{i-1} \leq 2^{k}$ for some $i \in [1, 2,
\ldots, \floor{\log B + 1}]$ or $T > 2^{k} = 2^{i - 1}$.

In case 1, the randomly chosen $i$ can only take on the following values: $i \in [1, 2, k + 1]$. Any greater value of
$i$ would surpass the bound $2^k$. Since we are choosing $i$ at random from $\floor{\log B + 1}$ values, the probability
of each $i$ would be $\dfrac{1}{\floor{\log B + 1}}$.




\end{document}
