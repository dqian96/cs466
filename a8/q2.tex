\documentclass{article}

\usepackage[utf8]{inputenc}
\usepackage{amsmath}
\usepackage{graphicx}
\usepackage{enumitem}
\usepackage[parfill]{parskip}
\graphicspath{ {.} }
\usepackage{listings}
\usepackage{xcolor}
\lstset { %
    language=Python,
    backgroundcolor=\color{black!5}, % set backgroundcolor
    basicstyle=\footnotesize,% basic font setting
}


\newtheorem{theorem}{Theorem}

\usepackage{algorithm}
\usepackage[noend]{algpseudocode}

\usepackage{mathtools}
\DeclarePairedDelimiter\floor{\lfloor}{\rfloor}


\title{CS 466: Algorithm Design and Analysis - Assignment 6}
\author{Qian, Yan Liang (David)}
\date{Date of submission: November 13, 2018}

\begin{document}
\newpage

\section{Question 2}

a)

We will prove that this deterministic algorithm has competitive ratio $\dfrac{1}{\sqrt{B}}$.

Consider case 1, $M \geq T$.  In this case, the algorithm will always accept a bid $D$ such that $T = \sqrt{B} \leq D
\leq M$ as such a bid will always appear in the sequence and the algorithm will accept the first one it sees. The
optimal algorithm would pick $M$. Then, the ratio between this algorithm and the OPT is $\dfrac{ALG}{OPT} =
\dfrac{D}{M}$. In the worst case, we minimize this ratio by considering the largest $M$ and smallest $D$ to get a lower
bound of $\dfrac{\sqrt{B}}{B} = \dfrac{1}{\sqrt{B}}$. \\

Consider case 2, $M < T$. In this case, the algorithm will pick the last bid $D$ such that $1 \leq D \leq M < T = \sqrt{B}$.
The optimal algorithm would pick $M < \sqrt{B}$. Then, the ratio between this algorithm and OPT is $\dfrac{ALG}{OPT} =
\dfrac{D}{M}$. In the worst case, we minimize the ratio by considering the largest $M$ and the smallest $D$ to get a
lower bound of $\dfrac{1}{\sqrt{B} - 1} > \dfrac{1}{\sqrt{B}}$. Take $\dfrac{1}{\sqrt{B}}$ as the lower bound. (Note
that in the case where $B = 1$, then $\dfrac{1}{\sqrt{B}} = 1$ still satisfies as the lower bound). \\

Thus, in both possible cases, we see that the ratio between the algorithm's performance to that of the optimal algorithm
is $\dfrac{ALG}{OPT} \geq \dfrac{1}{\sqrt{B}}$. Thus, $ALG \geq \dfrac{1}{\sqrt{B}} OPT$ and the algorithm is
$\dfrac{1}{\sqrt{B}}$-competitive. \\


b)

We will prove that the expected competitive ratio is at least $\dfrac{1}{2 \log B}$.

Consider any $M$ such that $2^k \leq M < 2^{k+1}$ for any choice of $k \in [1 - 1, 2 - 1, 3 - 1, \ldots, \floor{\log B +
1} - 1]$. This means there exists a $k$ in the stated range that satsifies any given $M$ in this problem. To calculate
the expected competitive ratio, we notice that $i$ is a random variable that determines the threshold.  The expectation
is thus the sum over all possible $i$ of the probability of choosing such $i$ multiplied by the competitive ratio given
such threshold $T = 2^{i - 1}$.

Let's look at the part of the sum where the choice of $i$ is such that $T = 2^{i-1} \leq 2^{k}$. To satisfy this bound,
$i \in [1, 2, k + 1]$ must be true. For any $i$ in that given range, we know that the bid accepted is $D \geq T = 2^{i - 1}$ and the
optimal bid is $M < 2^{k + 1}$. The competitive ratio is $\dfrac{ALG}{OPT} = \dfrac{D}{M}$. Let's establish a lower bound on the
competitive ratio by setting $D = 2^{i - 1}$ (worst case) and letting the denominator be a higher value than what is possible:
$\dfrac{D}{M} \geq \dfrac{T}{M}  > \dfrac{2^{i - 1}}{2^{k + 1}}$. This is a lower bound on the competitive ratio as a
function of $i$. Now, the probability of choosing each $i$ is $\dfrac{1}{\floor{\log B + 1}}$. The expected competitive
ratio is now $\mathbf{E}(\dfrac{ALG}{OPT}) \geq \dfrac{1}{\floor{\log B + 1}} \sum_{i = 1}^{k + 1} \dfrac{2^{i - 1}}{2^{k +
1}} + \beta$, where $\beta$ is the half of the expectation for when $i > k + 1$.

Now, let's look at the part of the expectation where the choice of $i$ is such that $T = 2^{i - 1} > 2^{k}$. Since the
choice of $i$ must be an integer, we have $i \in [k + 2, \ldots, \floor{\log B + 1}]$. However, for any choice of $i$ in
this range, we have $T = 2^{i - 1} \geq 2^{k + 1} > M$. Since the threshold is bigger than the maximum value in the
sequence, the algorithm doesn't bid and gets a reward of 0. This means that the competitve ratio is also 0 for these
choices of $i$. Thus, we see that $\beta = 0$ and the expected competitve ratio is $\mathbf{E}(\dfrac{ALG}{OPT}) \geq
\dfrac{1}{\floor{\log B + 1}} \sum_{i = 1}^{k + 1} \dfrac{2^{i - 1}}{2^{k + 1}}$.

Let's simplify this expression:

\begin{align*}
    \mathbf{E}(\dfrac{ALG}{OPT}) &\geq \dfrac{1}{\floor{\log B + 1}} \sum_{i = 1}^{k + 1} \dfrac{2^{i - 1}}{2^{k +1 }} \\
    &\geq \dfrac{1}{\floor{\log B + 1}} \dfrac{1}{2^{k + 2}} \sum_{i = 1}^{k + 1} 2^i \\
    &\geq \dfrac{1}{\floor{\log B + 1}} \dfrac{1}{2^{k + 2}} (2^{k + 2} - 2)\\
    &\geq \dfrac{1}{\floor{\log B + 1}} \dfrac{2^{k + 1} - 1}{2^{k + 1}} && \text{$1 - \dfrac{1}{2^{k +1}} \geq
        \dfrac{1}{2}, \forall k \geq 0$} \\
    &\geq \dfrac{1}{2\floor{\log B + 1}}
\end{align*}


Thus, we have proven that this randomized algorithm has an expected competitve ratio at least $\dfrac{1}{2 (\log B + 1)}$

\end{document}
